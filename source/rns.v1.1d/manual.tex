\documentstyle[11pt]{article}
\topmargin=-.5cm
\textheight=22cm
\oddsidemargin=.0cm
\textwidth=15.8cm
\baselineskip=18pt
\def\bee{\begin{equation}}
\def\ee{\end{equation}}
\def\<{\noindent }

\begin{document}

\title{\bf Users Manual for RNS}
\author{ \bf Nikolaos Stergioulas}
\date{\today}
\maketitle

\section{Introduction}

RNS is a code that constructs models of {\it rapidly rotating, relativistic, 
compact stars} using tabulated equations of state which are supplied by the 
user.
The code is written after the KEH method (Komatsu, Eriguchi \& Hachisu, 
1989) and modifications introduced by Cook, Shapiro \& Teukolsky (1994). 
It can compute individual models as well as sequences of 
fixed mass, rest mass, angular velocity or angular momentum models. All models
assume uniform rotation. You can read more about this code in 
Stergioulas \& Friedman (1994) and references therein.
 
\section{Compiling the code}

RNS is written in the ANSI C programming language. Machine specific 
commands were avoided,
so that the C program should compile on any platform using an ANSI C compiler.
The source code comes with a Unix {\tt Makefile}. Before running the 
{\tt Makefile} for the first time, one should check to see if it uses the 
correct compiler (the default is {\tt cc}) and the correct flags for choosing 
the ANSI C option (the default is {\tt -std1}), and optimization (default 
is {\tt -O})
 - if not, the {\tt Makefile} should be edited. Once the {\tt Makefile} is 
correct, run {\tt make} to obtain the executable {\tt rns}. If 
significant changes to the {\tt Makefile} are needed in order to compile it on 
your platform, please send the changes
to the author, so that they can be included in future versions.

\section{Supplying the equation of state}

RNS can compute stars with either a polytropic or tabulated 
equation of state (EOS). 
The format and units for each type of EOS is described in this 
section. 

\subsection{Tabulated Equation of State}
RNS needs a tabulated, zero-temperature equation of state (EOS), as an input, 
in order to run. 
You will have to format the EOS file in a specific way, as described here: 
The first line in the EOS file should contain the number of tabulated points. 
The remaining lines should consist of four columns - energy 
density (in ${\rm gr/cm^3}$), pressure (in ${\rm dynes/cm^2}$), enthalpy (in
${\rm cm^2/s^2}$), and baryon number density (in ${\rm cm^{-3}}$). The 
{\it enthalpy} is defined as

\bee H(P)=\int_0^P { c^2 dP \over (\epsilon +P)}, \ee

\<where $\epsilon$ is the energy density, $P$ is the pressure and $c$ is the
speed of light.

The number of points should be limited to 200. Example files 
(e.g. {\tt eosC}) are supplied with the source code. The EOS file needs to be 
specified
using the following flag:

\vspace{0.3cm}

{\tt rns -f} {\it filename}

\vspace{0.3cm}

A program ``HnG.c'' is included which will convert an EOS file 
tabulated in the form pressure v.s. energy density to the 
form required by RNS. See the program listing for more details.

\subsection{Polytropic Equations of State}

The default type of equation of state for RNS is tabulated, but RNS
will also compute polytropic stars. The polytropic equation of
state is 
\begin{equation}
	p = K \rho_0^{1 + 1/n}
\end{equation}
where $p$ is the pressure, $\rho_0$ is the baryon mass density,
$K$ is the polytropic constant, and $n$ is the polytropic index. To 
specify a polytrope with index $n$, the following command line
options need to be specified:

\vspace{0.3cm}

{\tt rns -q poly -N} {\it index}

\vspace{0.3cm}

The program uses the dimensionless units described by 
Cook, Shapiro \& Teukolsky 1994. All quantities are
reported in dimensionless units if a polytropic star
is indicated.


\section{Specifying model parameters and other options}

RNS has several options which allow you to specify model parameters and choose 
from different output formats. A model is defined uniquely by specifying two
parameters - one will always be the central energy density and the other
can be one of the following: mass, rest mass, angular velocity, 
angular momentum, or the ratio of the polar coordinate radius to the 
coordinate equatorial radius ({\it axes ratio}, see definition of coordinates 
later in the text) .

\vspace{0.3cm}
  
\<The parameters are specified using the following flags:
 
\begin{description}

\item{\tt -e} {\it central energy density} in ${\rm gr/cm^3}$

\item{\tt -r} {\it axes ratio}

\item{\tt -m} {\it mass} in ${\rm M}_{\odot}$

\item{\tt -z} {\it rest mass} in ${\rm M_{\odot}}$

\item{\tt -o} {\it angular velocity} in ${\rm 10^4 s^{-1}}$

\item{\tt -j} {\it angular momentum} in ${\rm GM_{\odot}^2/c}$
 
\end{description}




\< Note that if a polytropic star is requested, dimensionless 
units should be used. Consult the tables given in
Cook et. al 1994 to find suitable values for these parameters.

\vspace{0.2cm}

\<The code is written so as to directly construct a model when 
the axes ratio is specified (which is therefore the fastest option). If one 
specifies a different parameter e.g. mass, the code constructs several (usually 
more than ten) models by varying the axes ratio, until it finds a model, for 
which the chosen parameter is within some allowed tolerance of the specified 
value. The default tolerance is $10^{-4}$ (relative error) and can be changed 
by using the flag

\vspace{0.3cm}

{\tt -b} {\it tolerance}

\vspace{0.3cm}

\<A smaller tolerance means more models will be constructed.

\vspace{0.2cm}

A model is constructed by iteratively solving the field equations and the 
hydrostatic equilibrium equation, until the coordinate equatorial radius changes
by less than a specified relative accuracy. The default accuracy is $10^{-5}$
and this can be changed by using the flag

\vspace{0.3cm}

{\tt -a} {\it accuracy}
 
\vspace{0.3cm}

\<Convergence is monitored by printing out the relative difference in the 
coordinate equatorial radius from one iteration to the next. This print-out can 
be suppressed by using the flag

\vspace{0.3cm}

{\tt -d} 0

\vspace{0.3cm}

\<In rare cases, such as for unstable models of very stiff equations of state,
the iteration may not, at first, converge. Such cases can easily be fixed by 
using a relaxation factor in the iteration. A factor of 0.8 usually makes the
iteration convergent (the default is 1.0, which amounts to no relaxation). 
It can be specified using

\vspace{0.3cm}

{\tt -c} {\it relaxation factor}
 
\vspace{0.3cm}

\<Specifying the above parameters is not sufficient to get the program started.
One also needs to select the task that is to be performed.

\section{Tasks}

RNS can perform eight different tasks, which we list below, along with the
required options for each task:

\begin{description}
 
\item{{\tt -t model}  Computes a model with fixed central energy density 
                     $\epsilon_c$ and {\it axes ratio}. Requires {\tt -e}
                     and {\tt -r}} 

\item{{\tt -t gmass}  Computes a model with fixed $\epsilon_c$ and 
                     {\it gravitational mass}. Requires {\tt -e} and {\tt -m}} 

\item{{\tt -t rmass}  Computes a model with fixed $\epsilon_c$ and 
                     {\it rest mass}. Requires {\tt -e} and {\tt -z}} 

\item{{\tt -t omega}  Computes a model with fixed $\epsilon_c$ and 
                     {\it angular velocity}. Requires {\tt -e} and {\tt -o}} 
 
\item{{\tt -t jmoment}  Computes a model with fixed $\epsilon_c$ and 
                       {\it angular momentum}. Requires {\tt -e} and {\tt -j}} 

\item{{\tt -t static}  For a given $\epsilon_c$ computes the nonrotating model. 
                      Requires {\tt -e}}. 
 
\item{{\tt -t kepler}  For a given $\epsilon_c$ computes the model with 
                      Keplerian angular velocity. The default relative 
                      accuracy is 
                      $10^{-4}$ and can be changed with {\tt -b} {\it 
                      tolerance}. Requires {\tt -e}}. 
  
\item{\tt -t test}  Computes the test model.

\end{description}

\vspace{0.2cm}

\<To find out if the source code compiled correctly, you can run the test model
 
\vspace{0.2cm}

{\tt rns -f eosC -t test}

\vspace{0.2cm}

\<and compare the output to the file {\tt test.out}.

\vspace{0.2cm}

For specific examples of the above tasks and their output, see the
file {\tt examples.test}. 


\section{Sequences}

You can obtain a sequence of models with the same two parameters fixed 
(instead of just one model) by specifying a range of central energy 
densities and the number of models desired. For example

\vspace{0.2cm}

{\tt -e 1e15 -l 3e15 -n 10}

\vspace{0.2cm}

\<will give a sequence of ten models between the central energy densities
of $10^{15}$ and $3 \times 10^{15}$ ${\rm gr/cm^3}$. The models will be
equally spaced in $\log \epsilon_c$. You can do this with any of the
tasks described in the previous section. However, you have to make sure
that the models you requested actually exist. For example, if you want to 
compute a sequence of constant rest mass, you should first compute and 
examine the nonrotating and Keplerian sequences for a wide range of energy 
densities and then select a sub-range of $\epsilon_c$ for which stars of the 
given rest mass exist - otherwise, the code will fail to converge.
 
\section{Output}

RNS prints out 17 physical quantities upon succesfull computation of a model.
These are:

\begin{description}

\item{ $\epsilon_c$ {\it central energy density}}
\item{ $M$ {\it gravitational mass}}
\item{ $M_0$ {\it rest mass}}
\item{ $R_e$ {\it radius at the equator (circumferencial, i.e. $2 \pi R_e$ is 
the proper circumference) }}
\item{ $\Omega$ {\it angular velocity}}
\item{ $\Omega_p$ {\it angular velocity of a particle in circular orbit 
                      at the equator}}
\item{ $T/W$ {\it rotational/gravitational energy}}
\item{ $cJ/GM_{\odot}^2$ {\it angular momentum}}
\item{ $I$ {\it moment of inertia (except for nonrotating model)}}
\item{ $\Phi_2$ {\it quadrupole moment (program needs to be compiled
	on HIGH resolution for this to be accurate)}}
\item{ $h_+$ {\it height from surface of last stable co-rotating circular 
               orbit in equatorial plane (circumferencial) - if none, then all 
              such orbits are stable}}
\item{ $h_-$ {\it height from surface of last stable counter-rotating circular 
               orbit in equatorial plane (circumferential) - if none, then all 
               such orbits are stable}}
\item{ $Z_p$ {\it polar redshift}}
\item{ $Z_b$ {\it backward equatorial redshift}}
\item{ $Z_f$ {\it forward equatorial redshift}}
\item{ $\omega_c/ \Omega$ {\it ratio of central value of potential $\omega$ to 
                             $\Omega$}}
\item{ $r_e$ {\it coordinate equatorial radius }}
\item{ $r_p/r_e$ {\it axes ratio (polar to equatorial)}}

\end{description} 

\<The following values for the physical constants are used: $c=2.9979 \times
10^{10} {\rm cm/s^{-1}}$, $G=6.6732 \times 10^{-8} {\rm g^{-1} cm^3 s^2}$, 
$m_B=1.66 \times 10^{-24} {\rm gr}$, and $M_{\odot}=1.987 \times 10^{33}
{\rm gr}$.
The coordinates of the stationary, axisymetric spacetime used to model the 
compact star are defined through the metric

\bee ds^2=-e^{\gamma+ \rho} \ dt^2 + e^{2 \alpha} \bigl(dr^2 + r^2 d \theta^2 
          \bigr) + e^{\gamma - \rho} r^2 \sin^2 \theta \bigl( d \phi - \omega dt
          \bigr)^2. \ee

\<where the potentials $\gamma, \rho, \alpha$ and $\omega$ are functions of 
$r$ and 
$\theta$ only. The matter inside the neutron star is approximated by a perfect 
fluid.
 
\subsection{Printing Formats}

There are two different printing formats:

\begin{description}

\item{{\tt -p 1} is a detailed vertical list of the above quantities (default)}
\item{{\tt -p 2} is a compact horizontal print-out of the same quantities,
                 which is useful when computing sequences}.
\end{description}

\<In addition to the above quantities, one can print out the values of the
four metric potentials and the pressure at every grid point with

\vspace{0.2cm}
  
{\tt -p 3}

\vspace{0.2cm}

\<This is a very long print out and should be directed to a file rather than to
the screen. The format is:

\vspace{0.2cm}

 $s=r/(r+r_e)$ \hspace{0.6cm} $\cos \theta$ \hspace{0.6cm} $\rho$ \hspace{0.6cm} 
$\gamma$ \hspace{0.6cm} $\alpha$ \hspace{0.6cm} $\omega (10^4 {\rm s^{-1}})$ 
\hspace{0.6cm}  P $(dynes/cm^2)$

\section{Numerical Grid}

\<RNS uses a uniform 2-D grid in the variables $s=r/(r+r_e)$ and $\mu=
 \cos \theta$. The center is at $s=0$, the equator at $s=0.5$ and infinity at
$s=1$. The equatorial plane is at $\theta=\pi/2$ while the pole is at $\theta=0$.
If one uses $SDIV$ points in the $s$-direction and $MDIV$ points in the 
$\mu$-direction, then the grid points are obtained by the formulas

\bee s[i]=0.9999 \Bigl( {1-i \over SDIV-1} \Bigr), \ee

\bee \mu[j]=\Bigl( {1-j \over MDIV-1} \Bigr). \ee

\<The default grid size is $MDIV \times SDIV=65 \times 129$. You can change this 
by editing the {\tt Makefile}. If you specify too large a grid size and get a 
memory segmentation fault, you can fix this by issuing the command 

\vspace{0.2cm}

{\tt limit stacksize unlimited}

\vspace{0.2cm}

\<in a UNIX environment.

\section{Availability}

RNS is available as a public domain program and can be downloaded by anonymous
ftp from {\tt pauli.phys.uwm.edu} in the directory 
{\tt /pub/rns}. 
You can download either {\tt rns}.{\it version}.{\tt tar.Z} or 
{\tt rns}.{\it version}.{\tt tar.gz}. 
The code and documentation are also available at the website:

\vspace{0.2cm}

{\tt www.gravity.phys.uwm.edu/Code/rns}

\vspace{0.2cm}

\<Please send any suggestions, comments or
questions to the author at the following address:

\vspace{0.2cm}

\< \hspace{0.7cm} Nikolaos Stergioulas

\< \hspace{0.7cm} E-mail: niksterg@pauli.phys.uwm.edu 

\section{Acknowledgments}

I am gratefull to John L. Friedman for suggesting this project and for all his
advice and support. Thanks also go to Scott Koranda for making the source code
compile correctly on a SUN workstation. This project was supported by NSF
grant PHY-9105935.

\section{References}

Komatsu H., Eriguchi Y. \& Hachisu I., 1989, {\it MNRAS}, {\bf 237}, 355

\<Cook G.B., Shapiro S.L. \& Teukolsky S.A., 1994, {\it APJ}, {\bf 422}, 227

\<Stergioulas N. \& Friedman J.L., 1994, in press ({\it ApJ}, 1995, May 1)

  
\end{document}

